\documentclass{article}
\usepackage{pgf}
\usepackage{pgfpages}

% Options for packages loaded elsewhere
\PassOptionsToPackage{unicode}{hyperref}
\PassOptionsToPackage{hyphens}{url}
%

\usepackage{amsmath,amssymb}
\usepackage{lmodern}
\usepackage{iftex}
\ifPDFTeX
\usepackage[T1]{fontenc}
\usepackage[utf8]{inputenc}
\usepackage{pythontex}
\usepackage{textcomp} % provide euro and other symbols
\else % if luatex or xetex
\usepackage{unicode-math}
\defaultfontfeatures{Scale=MatchLowercase}
\defaultfontfeatures[\rmfamily]{Ligatures=TeX,Scale=1}
\fi
% Use upquote if available, for straight quotes in verbatim environments
\IfFileExists{upquote.sty}{\usepackage{upquote}}{}
\IfFileExists{microtype.sty}{% use microtype if available
	\usepackage[]{microtype}
	\UseMicrotypeSet[protrusion]{basicmath} % disable protrusion for tt fonts
}{}
\makeatletter
\@ifundefined{KOMAClassName}{% if non-KOMA class
	\IfFileExists{parskip.sty}{%
		\usepackage{parskip}
	}{% else
		\setlength{\parindent}{0pt}
		\setlength{\parskip}{6pt plus 2pt minus 1pt}}
}{% if KOMA class
	\KOMAoptions{parskip=half}}
\makeatother
\usepackage{xcolor}
\IfFileExists{xurl.sty}{\usepackage{xurl}}{} % add URL line breaks if available
\IfFileExists{bookmark.sty}{\usepackage{bookmark}}{\usepackage{hyperref}}
\hypersetup{
	hidelinks,
	pdfcreator={LaTeX via pandoc}}
\urlstyle{same} % disable monospaced font for URLs
\usepackage{color}
\usepackage{fancyvrb}
\newcommand{\VerbBar}{|}
\newcommand{\VERB}{\Verb[commandchars=\\\{\}]}
\DefineVerbatimEnvironment{Highlighting}{Verbatim}{commandchars=\\\{\}}
% Add ',fontsize=\small' for more characters per line
\newenvironment{Shaded}{}{}
\newcommand{\AlertTok}[1]{\textcolor[rgb]{1.00,0.00,0.00}{\textbf{#1}}}
\newcommand{\AnnotationTok}[1]{\textcolor[rgb]{0.38,0.63,0.69}{\textbf{\textit{#1}}}}
\newcommand{\AttributeTok}[1]{\textcolor[rgb]{0.49,0.56,0.16}{#1}}
\newcommand{\BaseNTok}[1]{\textcolor[rgb]{0.25,0.63,0.44}{#1}}
\newcommand{\BuiltInTok}[1]{#1}
\newcommand{\CharTok}[1]{\textcolor[rgb]{0.25,0.44,0.63}{#1}}
\newcommand{\CommentTok}[1]{\textcolor[rgb]{0.38,0.63,0.69}{\textit{#1}}}
\newcommand{\CommentVarTok}[1]{\textcolor[rgb]{0.38,0.63,0.69}{\textbf{\textit{#1}}}}
\newcommand{\ConstantTok}[1]{\textcolor[rgb]{0.53,0.00,0.00}{#1}}
\newcommand{\ControlFlowTok}[1]{\textcolor[rgb]{0.00,0.44,0.13}{\textbf{#1}}}
\newcommand{\DataTypeTok}[1]{\textcolor[rgb]{0.56,0.13,0.00}{#1}}
\newcommand{\DecValTok}[1]{\textcolor[rgb]{0.25,0.63,0.44}{#1}}
\newcommand{\DocumentationTok}[1]{\textcolor[rgb]{0.73,0.13,0.13}{\textit{#1}}}
\newcommand{\ErrorTok}[1]{\textcolor[rgb]{1.00,0.00,0.00}{\textbf{#1}}}
\newcommand{\ExtensionTok}[1]{#1}
\newcommand{\FloatTok}[1]{\textcolor[rgb]{0.25,0.63,0.44}{#1}}
\newcommand{\FunctionTok}[1]{\textcolor[rgb]{0.02,0.16,0.49}{#1}}
\newcommand{\ImportTok}[1]{#1}
\newcommand{\InformationTok}[1]{\textcolor[rgb]{0.38,0.63,0.69}{\textbf{\textit{#1}}}}
\newcommand{\KeywordTok}[1]{\textcolor[rgb]{0.00,0.44,0.13}{\textbf{#1}}}
\newcommand{\NormalTok}[1]{#1}
\newcommand{\OperatorTok}[1]{\textcolor[rgb]{0.40,0.40,0.40}{#1}}
\newcommand{\OtherTok}[1]{\textcolor[rgb]{0.00,0.44,0.13}{#1}}
\newcommand{\PreprocessorTok}[1]{\textcolor[rgb]{0.74,0.48,0.00}{#1}}
\newcommand{\RegionMarkerTok}[1]{#1}
\newcommand{\SpecialCharTok}[1]{\textcolor[rgb]{0.25,0.44,0.63}{#1}}
\newcommand{\SpecialStringTok}[1]{\textcolor[rgb]{0.73,0.40,0.53}{#1}}
\newcommand{\StringTok}[1]{\textcolor[rgb]{0.25,0.44,0.63}{#1}}
\newcommand{\VariableTok}[1]{\textcolor[rgb]{0.10,0.09,0.49}{#1}}
\newcommand{\VerbatimStringTok}[1]{\textcolor[rgb]{0.25,0.44,0.63}{#1}}
\newcommand{\WarningTok}[1]{\textcolor[rgb]{0.38,0.63,0.69}{\textbf{\textit{#1}}}}
\setlength{\emergencystretch}{3em} % prevent overfull lines
\providecommand{\tightlist}{%
	\setlength{\itemsep}{0pt}\setlength{\parskip}{0pt}}
\setcounter{secnumdepth}{-\maxdimen} % remove section numbering
\ifLuaTeX
\usepackage{selnolig}  % disable illegal ligatures
\fi


\pgfpagesdeclarelayout{boxed}
{
	\edef\pgfpageoptionborder{0pt}
}
{
	\pgfpagesphysicalpageoptions
	{%
		logical pages=1,%
	}
	\pgfpageslogicalpageoptions{1}
	{
		border code=\pgfsetlinewidth{2pt}\pgfstroke,%
		border shrink=\pgfpageoptionborder,%
		resized width=.95\pgfphysicalwidth,%
		resized height=.95\pgfphysicalheight,%
		center=\pgfpoint{.5\pgfphysicalwidth}{.5\pgfphysicalheight}%
	}%
}



\pgfpagesuselayout{boxed}
\usepackage{url}
\usepackage{authblk}
\usepackage{amsmath}
\usepackage{setspace}\doublespacing
\usepackage{graphicx} 
\usepackage{amssymb}


\usepackage{amsfonts}
\usepackage{amssymb}
\usepackage{floatflt}
\usepackage{lipsum}
%\usepackage[demo]{graphicx}
\usepackage{upquote} % Upright quotes for verbatim code
\usepackage{eurosym} % defines \euro
\usepackage[mathletters]{ucs} % Extended unicode (utf-8) support
\usepackage[utf8x]{inputenc} % Allow utf-8 characters in the tex document
\usepackage{fancyvrb} % verbatim replacement that allows latex
\usepackage{grffile} % extends the file name processing of package graphics 
\usepackage{xepersian}

\DefineVerbatimEnvironment{Highlighting}{Verbatim}{commandchars=\\\{\}}
% Pygments definitions

\makeatletter
\def\PY@reset{\let\PY@it=\relax \let\PY@bf=\relax%
	\let\PY@ul=\relax \let\PY@tc=\relax%
	\let\PY@bc=\relax \let\PY@ff=\relax}
\def\PY@tok#1{\csname PY@tok@#1\endcsname}
\def\PY@toks#1+{\ifx\relax#1\empty\else%
	\PY@tok{#1}\expandafter\PY@toks\fi}
\def\PY@do#1{\PY@bc{\PY@tc{\PY@ul{%
				\PY@it{\PY@bf{\PY@ff{#1}}}}}}}
\def\PY#1#2{\PY@reset\PY@toks#1+\relax+\PY@do{#2}}

\expandafter\def\csname PY@tok@w\endcsname{\def\PY@tc##1{\textcolor[rgb]{0.73,0.73,0.73}{##1}}}
\expandafter\def\csname PY@tok@c\endcsname{\let\PY@it=\textit\def\PY@tc##1{\textcolor[rgb]{0.25,0.50,0.50}{##1}}}
\expandafter\def\csname PY@tok@cp\endcsname{\def\PY@tc##1{\textcolor[rgb]{0.74,0.48,0.00}{##1}}}
\expandafter\def\csname PY@tok@k\endcsname{\let\PY@bf=\textbf\def\PY@tc##1{\textcolor[rgb]{0.00,0.50,0.00}{##1}}}
\expandafter\def\csname PY@tok@kp\endcsname{\def\PY@tc##1{\textcolor[rgb]{0.00,0.50,0.00}{##1}}}
\expandafter\def\csname PY@tok@kt\endcsname{\def\PY@tc##1{\textcolor[rgb]{0.69,0.00,0.25}{##1}}}
\expandafter\def\csname PY@tok@o\endcsname{\def\PY@tc##1{\textcolor[rgb]{0.40,0.40,0.40}{##1}}}
\expandafter\def\csname PY@tok@ow\endcsname{\let\PY@bf=\textbf\def\PY@tc##1{\textcolor[rgb]{0.67,0.13,1.00}{##1}}}
\expandafter\def\csname PY@tok@nb\endcsname{\def\PY@tc##1{\textcolor[rgb]{0.00,0.50,0.00}{##1}}}
\expandafter\def\csname PY@tok@nf\endcsname{\def\PY@tc##1{\textcolor[rgb]{0.00,0.00,1.00}{##1}}}
\expandafter\def\csname PY@tok@nc\endcsname{\let\PY@bf=\textbf\def\PY@tc##1{\textcolor[rgb]{0.00,0.00,1.00}{##1}}}
\expandafter\def\csname PY@tok@nn\endcsname{\let\PY@bf=\textbf\def\PY@tc##1{\textcolor[rgb]{0.00,0.00,1.00}{##1}}}
\expandafter\def\csname PY@tok@ne\endcsname{\let\PY@bf=\textbf\def\PY@tc##1{\textcolor[rgb]{0.82,0.25,0.23}{##1}}}
\expandafter\def\csname PY@tok@nv\endcsname{\def\PY@tc##1{\textcolor[rgb]{0.10,0.09,0.49}{##1}}}
\expandafter\def\csname PY@tok@no\endcsname{\def\PY@tc##1{\textcolor[rgb]{0.53,0.00,0.00}{##1}}}
\expandafter\def\csname PY@tok@nl\endcsname{\def\PY@tc##1{\textcolor[rgb]{0.63,0.63,0.00}{##1}}}
\expandafter\def\csname PY@tok@ni\endcsname{\let\PY@bf=\textbf\def\PY@tc##1{\textcolor[rgb]{0.60,0.60,0.60}{##1}}}
\expandafter\def\csname PY@tok@na\endcsname{\def\PY@tc##1{\textcolor[rgb]{0.49,0.56,0.16}{##1}}}
\expandafter\def\csname PY@tok@nt\endcsname{\let\PY@bf=\textbf\def\PY@tc##1{\textcolor[rgb]{0.00,0.50,0.00}{##1}}}
\expandafter\def\csname PY@tok@nd\endcsname{\def\PY@tc##1{\textcolor[rgb]{0.67,0.13,1.00}{##1}}}
\expandafter\def\csname PY@tok@s\endcsname{\def\PY@tc##1{\textcolor[rgb]{0.73,0.13,0.13}{##1}}}
\expandafter\def\csname PY@tok@sd\endcsname{\let\PY@it=\textit\def\PY@tc##1{\textcolor[rgb]{0.73,0.13,0.13}{##1}}}
\expandafter\def\csname PY@tok@si\endcsname{\let\PY@bf=\textbf\def\PY@tc##1{\textcolor[rgb]{0.73,0.40,0.53}{##1}}}
\expandafter\def\csname PY@tok@se\endcsname{\let\PY@bf=\textbf\def\PY@tc##1{\textcolor[rgb]{0.73,0.40,0.13}{##1}}}
\expandafter\def\csname PY@tok@sr\endcsname{\def\PY@tc##1{\textcolor[rgb]{0.73,0.40,0.53}{##1}}}
\expandafter\def\csname PY@tok@ss\endcsname{\def\PY@tc##1{\textcolor[rgb]{0.10,0.09,0.49}{##1}}}
\expandafter\def\csname PY@tok@sx\endcsname{\def\PY@tc##1{\textcolor[rgb]{0.00,0.50,0.00}{##1}}}
\expandafter\def\csname PY@tok@m\endcsname{\def\PY@tc##1{\textcolor[rgb]{0.40,0.40,0.40}{##1}}}
\expandafter\def\csname PY@tok@gh\endcsname{\let\PY@bf=\textbf\def\PY@tc##1{\textcolor[rgb]{0.00,0.00,0.50}{##1}}}
\expandafter\def\csname PY@tok@gu\endcsname{\let\PY@bf=\textbf\def\PY@tc##1{\textcolor[rgb]{0.50,0.00,0.50}{##1}}}
\expandafter\def\csname PY@tok@gd\endcsname{\def\PY@tc##1{\textcolor[rgb]{0.63,0.00,0.00}{##1}}}
\expandafter\def\csname PY@tok@gi\endcsname{\def\PY@tc##1{\textcolor[rgb]{0.00,0.63,0.00}{##1}}}
\expandafter\def\csname PY@tok@gr\endcsname{\def\PY@tc##1{\textcolor[rgb]{1.00,0.00,0.00}{##1}}}
\expandafter\def\csname PY@tok@ge\endcsname{\let\PY@it=\textit}
\expandafter\def\csname PY@tok@gs\endcsname{\let\PY@bf=\textbf}
\expandafter\def\csname PY@tok@gp\endcsname{\let\PY@bf=\textbf\def\PY@tc##1{\textcolor[rgb]{0.00,0.00,0.50}{##1}}}
\expandafter\def\csname PY@tok@go\endcsname{\def\PY@tc##1{\textcolor[rgb]{0.53,0.53,0.53}{##1}}}
\expandafter\def\csname PY@tok@gt\endcsname{\def\PY@tc##1{\textcolor[rgb]{0.00,0.27,0.87}{##1}}}
\expandafter\def\csname PY@tok@err\endcsname{\def\PY@bc##1{\setlength{\fboxsep}{0pt}\fcolorbox[rgb]{1.00,0.00,0.00}{1,1,1}{\strut ##1}}}
\expandafter\def\csname PY@tok@kc\endcsname{\let\PY@bf=\textbf\def\PY@tc##1{\textcolor[rgb]{0.00,0.50,0.00}{##1}}}
\expandafter\def\csname PY@tok@kd\endcsname{\let\PY@bf=\textbf\def\PY@tc##1{\textcolor[rgb]{0.00,0.50,0.00}{##1}}}
\expandafter\def\csname PY@tok@kn\endcsname{\let\PY@bf=\textbf\def\PY@tc##1{\textcolor[rgb]{0.00,0.50,0.00}{##1}}}
\expandafter\def\csname PY@tok@kr\endcsname{\let\PY@bf=\textbf\def\PY@tc##1{\textcolor[rgb]{0.00,0.50,0.00}{##1}}}
\expandafter\def\csname PY@tok@bp\endcsname{\def\PY@tc##1{\textcolor[rgb]{0.00,0.50,0.00}{##1}}}
\expandafter\def\csname PY@tok@fm\endcsname{\def\PY@tc##1{\textcolor[rgb]{0.00,0.00,1.00}{##1}}}
\expandafter\def\csname PY@tok@vc\endcsname{\def\PY@tc##1{\textcolor[rgb]{0.10,0.09,0.49}{##1}}}
\expandafter\def\csname PY@tok@vg\endcsname{\def\PY@tc##1{\textcolor[rgb]{0.10,0.09,0.49}{##1}}}
\expandafter\def\csname PY@tok@vi\endcsname{\def\PY@tc##1{\textcolor[rgb]{0.10,0.09,0.49}{##1}}}
\expandafter\def\csname PY@tok@vm\endcsname{\def\PY@tc##1{\textcolor[rgb]{0.10,0.09,0.49}{##1}}}
\expandafter\def\csname PY@tok@sa\endcsname{\def\PY@tc##1{\textcolor[rgb]{0.73,0.13,0.13}{##1}}}
\expandafter\def\csname PY@tok@sb\endcsname{\def\PY@tc##1{\textcolor[rgb]{0.73,0.13,0.13}{##1}}}
\expandafter\def\csname PY@tok@sc\endcsname{\def\PY@tc##1{\textcolor[rgb]{0.73,0.13,0.13}{##1}}}
\expandafter\def\csname PY@tok@dl\endcsname{\def\PY@tc##1{\textcolor[rgb]{0.73,0.13,0.13}{##1}}}
\expandafter\def\csname PY@tok@s2\endcsname{\def\PY@tc##1{\textcolor[rgb]{0.73,0.13,0.13}{##1}}}
\expandafter\def\csname PY@tok@sh\endcsname{\def\PY@tc##1{\textcolor[rgb]{0.73,0.13,0.13}{##1}}}
\expandafter\def\csname PY@tok@s1\endcsname{\def\PY@tc##1{\textcolor[rgb]{0.73,0.13,0.13}{##1}}}
\expandafter\def\csname PY@tok@mb\endcsname{\def\PY@tc##1{\textcolor[rgb]{0.40,0.40,0.40}{##1}}}
\expandafter\def\csname PY@tok@mf\endcsname{\def\PY@tc##1{\textcolor[rgb]{0.40,0.40,0.40}{##1}}}
\expandafter\def\csname PY@tok@mh\endcsname{\def\PY@tc##1{\textcolor[rgb]{0.40,0.40,0.40}{##1}}}
\expandafter\def\csname PY@tok@mi\endcsname{\def\PY@tc##1{\textcolor[rgb]{0.40,0.40,0.40}{##1}}}
\expandafter\def\csname PY@tok@il\endcsname{\def\PY@tc##1{\textcolor[rgb]{0.40,0.40,0.40}{##1}}}
\expandafter\def\csname PY@tok@mo\endcsname{\def\PY@tc##1{\textcolor[rgb]{0.40,0.40,0.40}{##1}}}
\expandafter\def\csname PY@tok@ch\endcsname{\let\PY@it=\textit\def\PY@tc##1{\textcolor[rgb]{0.25,0.50,0.50}{##1}}}
\expandafter\def\csname PY@tok@cm\endcsname{\let\PY@it=\textit\def\PY@tc##1{\textcolor[rgb]{0.25,0.50,0.50}{##1}}}
\expandafter\def\csname PY@tok@cpf\endcsname{\let\PY@it=\textit\def\PY@tc##1{\textcolor[rgb]{0.25,0.50,0.50}{##1}}}
\expandafter\def\csname PY@tok@c1\endcsname{\let\PY@it=\textit\def\PY@tc##1{\textcolor[rgb]{0.25,0.50,0.50}{##1}}}
\expandafter\def\csname PY@tok@cs\endcsname{\let\PY@it=\textit\def\PY@tc##1{\textcolor[rgb]{0.25,0.50,0.50}{##1}}}

\def\PYZbs{\char`\\}
\def\PYZus{\char`\_}
\def\PYZob{\char`\{}
\def\PYZcb{\char`\}}
\def\PYZca{\char`\^}
\def\PYZam{\char`\&}
\def\PYZlt{\char`\<}
\def\PYZgt{\char`\>}
\def\PYZsh{\char`\#}
\def\PYZpc{\char`\%}
\def\PYZdl{\char`\$}
\def\PYZhy{\char`\-}
\def\PYZsq{\char`\'}
\def\PYZdq{\char`\"}
\def\PYZti{\char`\~}
% for compatibility with earlier versions
\def\PYZat{@}
\def\PYZlb{[}
\def\PYZrb{]}
\makeatother

% Exact colors from NB
\definecolor{incolor}{rgb}{0.0, 0.0, 0.5}
\definecolor{outcolor}{rgb}{0.545, 0.0, 0.0}
\renewcommand\Authsep{، }
\renewcommand\Authands{ و }
\renewcommand\Authand{ و }

\newcommand{\namen}{نام و نام خانوادگی :}
\newcommand{\term}{Spring 2012}
\newcommand{\id}{شماره دانشجویی :}
\newcommand{\examdate}{3/28/12}
\newcommand{\timelimit}{50 Minutes}


\pagestyle{empty}
\addtolength{\textwidth}{7.5cm}
\addtolength{\textheight}{8cm}
\addtolength{\topmargin}{-3.5cm}
\addtolength{\oddsidemargin}{-4cm}
%\addtolength{\evensidemargin}{2cm}
\settextfont[Scale=1]{XB Zar}
\setlatintextfont{Liberation Mono}
%\def\LOGO{%
	%\begin{picture}(0,0)\unitlength=1cm
	%\put (0,-1) {\includegraphics[width=4.9em]{index.png}}
	%\end{picture}
	%}

\begin{document}
	
	\title {ویژگی‌های نیم رساناها: باتری‌ خورشیدی\\
		\vspace{0.5cm}
		\large گروه یک: سعید شیرانی، آبتین الماسی، امیرسهیل بلوچستان‌زاده\\}
	
	\author{نگارنده: سعید شیرانی} 
	
	\affil{ }
	\maketitle
	\newpage{}
	\normalsize 
	
	\section{هدف آزمایش}
	\vspace{5mm}
	\normalsize
	
	بررسی تغییرات جریان اتصال کوتاه ($I_{sc}$) با شدت نور‌فرودی، محاسبه‌ی ولتاژ مدار باز($V_{oc}$)، محاسبه‌ی سازه‌ی پرشدگی($FF$)بررسی تغییرات جریان اتصال کوتاه($I_{sc}$) با زاویه‌‌ی فرود(زاویه‌ی میان‌خط عمود برسطح باتری خورشیدی و پرتوی‌فرودی)
	
	\section{ابزار آزمایش:}
	باتری خورشید، ولت‌سنج، آمپرسنج، جعبه مقاومت، چراغ شش ولتی، میزچه مدرج
	\section{چگونگی انجام آزمایش:}
	
	نور چشمه را به گونه‌ای یکنواخت روی باتری خورشیدی بیندازید. \\
	1. جریان  اتصال کوتاه $I_{sc}$ را با بستن آمپرسنج به دوسر باتری خورشیدی(بی‌مقاومت) برحسب فاصله‌ی چشمه از آن اندازه‌بگیرید و در جدول زیر یادداشت کنید. منحنی جریان اتصال کوتاه $I_{sc}$، برحسب فاصله‌ی چشمه از باتری خورشیدی رسم کنید. برای خطی شدن نمودار می‌توان از شدت برحسب یک توان مناسب از فاصله رسم کنید تا رابطه خطی بدست آید. \\
	\textbf{نکته:} با تغییر فاصله‌ی چشمه از باتری خورشیدی، شدت نور تابیده به باتری تغییر می‌کند.
	\vspace{3cm}
	\begin{center}
		\begin{table}[h!]
			\centering
			
			\setlength{\tabcolsep}{20pt}
			\renewcommand{\arraystretch}{2}
			
			\begin{tabular}{|c|c|}
				\hline
				\centering
				$I_{sc} (mA)$ & \rl{ فاصله(cm)  \pm $0.1$ cm}\\
				\hline
				\hline
				$0.105\pm0.004$ & $90$ \\ 
				\hline
				$0.118\pm0.002$ & $85$ \\
				\hline
				$0.128\pm0.002$ & $80$ \\
				\hline
				$0.142\pm0.001$ & $75$ \\
				\hline
				$0.160\pm0.001$ & $70$ \\
				\hline
				$0.180\pm0.001$ & $65$ \\
				\hline
				$0.203\pm0.001$ & $60$ \\
				\hline
				$0.236\pm0.001$ & $55$ \\
				\hline
				$0.275\pm0.001$ & $50$ \\
				\hline
				$0.325\pm0.001$ & $45$ \\
				\hline
			\end{tabular}
			\caption{تغییرات جریان اتصال کوتاه برحسب فاصله‌ی چشمه از باتری خورشیدی}
		\end{table}
	\end{center}
	
	\newpage
	
\begin{latin}
% https://docs.google.com/spreadsheets/d/1amrrvDlvlMtcONc3FyaNBzunMMU_KPnoqfpJRYUN3aM/edit#gid=0
\end{latin}	
	
	2. مدار آزمایش را مانند شکل زیر ببندید.
	
	محل قرار گیری شکل
	
	
	باتری خورشیدی را در فاصله‌ی ثابت 40 سانتی‌متری قرار می‌دهیم. مقدار جعبه‌ی مقاومت را به صورت لگاریتمی(1, 2, 5, 10, 20, 50, 100, 200, 500, 1000, 2000, 5000) تغییر می‌دهیم تا حاصل ضرب IV بیشینه شود و مقادیر متناظر را در جدول 2 یادداشت می‌کنیم. مقاومت را تاجایی بالا می‌بریم ک دیگر مقدار V تغییری نکند. این $V_{oc}$ است. اینک پیرامون مقاومتی ‌که IV بیشینه است، برای چند مقاومت دیگر (کوچکتر و بزرگتر) I و V را اندازه می‌گیریم. بیشینه‌ی $I*V$ را مشخص می‌کنیم. این مقادیر همان $V_{mpp}$ و $I_{mpp}$ است. نمودار$I-V$برحسب $R$ را رسم کنید و یک منحنی مناسب در آن برازش کنید.
	
	
	
	\begin{latin}
		\vspace{3cm}
		\begin{center}
			\begin{table}[h!]
				\centering
				
				\setlength{\tabcolsep}{20pt}
				\renewcommand{\arraystretch}{2}
				
				\begin{tabular}{|c|c|c|c|}
					\hline
					\multicolumn{4}{|c|}{Changes in current and voltage according to the change in resistance in the resistance box} \\
					\hline
					R($\Omega$) & I(mA) & V & $I*v$\\
					\hline
					1&$0.40\pm0.01$ &$0.60mV\pm0.1mV$& $90$ \\ 
					\hline
					2&$0.40\pm0.01$ &$1.00mV\pm0.1mV$& $85$ \\
					\hline
					5&$0.40\pm0.01$ &$2.40mV\pm0.1mV$& $80$ \\
					\hline
					10&$0.40\pm0.01$ &$4.10mV\pm0.1mV$& $75$ \\
					\hline
					20&$0.40\pm0.01$ &$8.20mV\pm0.1mV$& $70$ \\
					\hline
					50&$0.39\pm0.01$ &$20.60mV\pm0.2mV$& $65$ \\
					\hline
					100&$0.39\pm0.01$ &$7.20mV\pm0.1mV$& $60$ \\
					\hline
					200&$0.39\pm0.01$ &$46.8mV\pm0.5mV$& $55$ \\
					\hline
					500&$0.38\pm0.01$ &$166.0mV\pm0.5mV$& $50$ \\
					\hline
					1000&$0.39\pm0.01$ &$0.39V\pm0.01V$& $45$ \\
					\hline
					2000&$0.39\pm0.01$ &$0.78V\pm0.01V$& $45$ \\
					\hline
					5000&$0.36\pm0.01$ &$1.81V\pm0.01V$& $45$ \\
					\hline
					10000&$0.28\pm0.01$ &$2.82V\pm0.01V$& $45$ \\
					\hline
					20000&$0.16\pm0.01$ &$3.43V\pm0.01V$& $45$ \\
					\hline
					50000&$0.07\pm0.01$ &$3.71V\pm0.01V$& $45$ \\
					\hline
					100000&$0.03\pm0.01$ &$3.80V\pm0.01V$& $45$ \\
					\hline
					200000&$0.02\pm0.01$ &$3.85V\pm0.01V$& $45$ \\
					\hline
					500000&$0.01\pm0.01$ &$3.87V\pm0.01V$& $45$ \\
					\hline
				\end{tabular}
				\caption{}
			\end{table}
		\end{center}
	\end{latin}
	\newpage
	منحنی $I-V$ را رسم ‌می‌کنیم و برای نمایش بهتر داده‌ها از محور لگاریتمی استفاده‌می کنیم.
	
	\newpage
	3.با توجه به اندازه‌گیری‌های بخش‌های پیشین برای $I_{sc}$ و $V_{oc}$ و $I_{mpp}$ و $V_{mpp}$ سازه‌ی پرشدگی و خطای آن را محاسبه کنید. 
	\newpage
	4.با کمک یک میزچه‌ی مدرج تغییرات $I$ را به صورت تابعی از زاویه‌ی فرود در فاصله‌ی ثابتی از لامپ اندازه‌گیری کنید. ونمودار تغییرات $I$ بر حسب $\cos^2 {\theta}$ را رسم کنید. آیا رابطه خطی است؟ علت آن را بینویسید.
	
	
	خطای تتا چند درجه است؟؟؟؟؟؟
	
	
	
	
	\begin{latin}
		\vspace{3cm}
		\begin{center}
			\begin{table}[h!]
				\centering
				
				\setlength{\tabcolsep}{8pt}
				\renewcommand{\arraystretch}{2}
				
				\begin{tabular}{|c|c|c|c|c|c|c|c|c|c|c|}
					\hline
					\multicolumn{11}{|c|}{Changes in current according to the change of landing angle} \\
					\hline
					$\theta$&0&10&20&30&40&50&60&70&80&90\\
					\hline
					$I\pm0.001mA$&0.086&0.083&0.079&0.077&0.068&0.059&0.050&0.038&0.021$\pm$0.003mA&0.013$\pm$0.003mA\\
					\hline
				\end{tabular}
				\caption{}
			\end{table}
		\end{center}
	\end{latin}
	\newpage
	
\section{پرسش‌ها}
1. چگونگی ساخت نیم‌رسانا‌های گونه‌ی n و گونه‌ی p را شرح دهید.\\
2. مراحل تبدیل انرژی نورانی به انرژی الکتریکی را در یک باتری خورشیدی توضیح دهید.\\
3. خطاهای موجود در آژمایش را بیان کنید و در صورت امکان راه حلی برای کاهش آن‌ها بیابید.\\
4. هم‌ارزی دو تعریف داده شده بریا جریان اتصال کوتاه را نشان دهید.\\
5. آیا اندازه‌گیری جریان اتصال کوتاه در بخش اول آزمایش این جریان را به درستی نشان می‌دهد؟ دلیل آن را بیان کنید.\\
6. آیا ولتاژ مدار بازی که به دست می‌آورید با تعریف نظری آن هم خوانی دارد؟ چرا؟\\
	
	
	
	
	
	
	
	
	
	
	
\end{document}